\documentclass[a4paper,11pt]{article}
\usepackage[utf8]{inputenc}
\usepackage{graphicx}
\usepackage[english]{babel}
\usepackage[vmargin=3.5cm, top=2cm]{geometry}
\usepackage[linktocpage=true]{hyperref}
\usepackage{enumitem}
\usepackage{longtable}
\usepackage{pdfpages}
\usepackage{float}
\usepackage{hyperref}
\usepackage[section]{placeins}
\hypersetup{
   colorlinks,
   citecolor=black,
   filecolor=black,
   linkcolor=black,
   urlcolor=black
}

\begin{document}

\begin{titlepage}

\centering \parindent=0pt
\newcommand{\HRule}{\rule{\textwidth}{1mm}}
\vspace*{\stretch{1}} \HRule\\[1cm]\Huge\bfseries
Graphics Programming Report\\[0.7cm]
\HRule\\[4cm]  
\large by 
\\Alexander Kirk
\\ and Mikkel Stolborg
\vspace*{\stretch{2}} \normalsize %
\begin{flushleft}
IT University of Copenhagen \\
GRPRP, S2015\\
Dan Lessin\\
\today \end{flushleft}
\end{titlepage}

\tableofcontents
\pagebreak
\section{Introduction}
For this project we decided to create a simulation of the solar system with the planets acting on each other using gravitational forces. The idea was to have the sun act as a point light and the planets move only with a set starting velocity, having the forces move them accordingly and the lightning of the sun ensure the planets are lit accordingly.
For more visuals we implemented camera controls so we could move around the solar system and to add to the simulation and visualise the forces of the planets we implemented a comet system to hurl comets through the system which would be affected by the gravitational wells of the planets.

First we will go over the structure of the completed works throughout the project, detailing the implementation of the system in its parts. 
After the weekly break down, we will present the results of the simulation, the workings of the different mechanics and their interplay. 
Lastly we will conclude upon the project detailing what we have done and what could have been done differently.
\include{"WeekStructure"}
\include{"Results"}
\include{"Conclusion"} 
\end{document}