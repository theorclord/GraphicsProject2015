\documentclass[a4paper,11pt]{article}
\usepackage[utf8]{inputenc}
\usepackage{graphicx}
\usepackage[english]{babel}
\usepackage[vmargin=3.5cm, top=2cm]{geometry}
\usepackage[linktocpage=true]{hyperref}
\usepackage{enumitem}
\usepackage{longtable}
\usepackage{pdfpages}
\usepackage{float}
\usepackage{hyperref}
\usepackage[section]{placeins}
\hypersetup{
   colorlinks,
   citecolor=black,
   filecolor=black,
   linkcolor=black,
   urlcolor=black
}

\begin{document}

\begin{titlepage}

\centering \parindent=0pt
\newcommand{\HRule}{\rule{\textwidth}{1mm}}
\vspace*{\stretch{1}} \HRule\\[1cm]\Huge\bfseries
Graphics Programming Report\\[0.7cm]
\HRule\\[4cm]  
\large by 
\\Alexander Kirk
\\ and Mikkel Stolborg
\vspace*{\stretch{2}} \normalsize %
\begin{flushleft}
IT University of Copenhagen \\
GRPRP, S2015\\
Dan Lessin\\
\today \end{flushleft}
\end{titlepage}

\tableofcontents
\pagebreak
\section{Introduction}
Solar system simulation choice.
For this project we decided to create a simulation of the solar system with the planets acting on each other using gravitational forces. The idea was to have the sun act as a point light and the planets move only with a set starting velocity, having the forces move them accordingly and the lightning of the sun ensure the planets are lit accordingly.

\include{"WeekStructure"}
\include{"Results"}
\include{"Conclusion"} 
\end{document}